%% example LaTeX script including macro definitions and parameter tables generated from a parameter file
%% (Tom Tetzlaff, 2022)

\documentclass[10pt,a4paper,american]{article}
\usepackage{geometry}
\geometry{verbose,a4paper,tmargin=2cm,bmargin=2cm,lmargin=2cm,rmargin=2cm}
\usepackage{amsmath}
\usepackage[table]{xcolor}
\renewcommand\familydefault{cmss}

\begin{document}

\section{Macros for parameter names}

%% macros definitions
\input{macros.tex}   %% generated by example.py from params.json


The following text examples use LaTeX macros for parameter names automatically generated from \texttt{params.json}:
\begin{itemize}
\item The size \PN (\verb+\PN+) of the network was chosen such that everybody is impressed.
\item Application of TTX is micked by setting the spike threshold \PVth (\verb+\PVth+) to infinity.
\item The weight $\PJI=-\Pg\PJ$ (\verb+\PJI+) of inhibitory synapses plays no role.
\end{itemize}
%%
To get an overview of all macro definitions, one can print this out in a table such Table \ref{tab:macros_table}. The table is automatically generated from the parameter dictionary \texttt{params.json} using the configuration defined in \texttt{config.yml}.
\begin{table}[ht!]
\begin{center}
  \parbox{0.8\linewidth}{       %% use parbox to define table width within latex
    \small%
    \centering%
    \renewcommand{\arraystretch}{1.2}%
    \noindent%
    \input{macros_table.tex} %% generated by example.py from params.json
    \caption{Macro definitions.}
    \label{tab:macros_table}
  }
\end{center}
\end{table}

\section{Parameter table}
The following parameter Table \ref{tab:parameter_table} is automatically generated from the parameter dictionary \texttt{params.json} using the configuration defined in \texttt{config.yml}: 
\begin{table}[ht!]
\begin{center}
  \parbox{0.8\linewidth}{       %% use parbox to define table width within latex
    \small%
    \centering%
    \renewcommand{\arraystretch}{1.2}%
    \noindent%
    \input{parameter_table.tex} %% generated by example.py from params.json
    \caption{Model and simulation parameters. Secondary parameters derived from primary parameters are marked in gray.}
    \label{tab:parameter_table}
  }
\end{center}
\end{table}

\end{document}
 